\chapter{Blinking LEDs}
This work is inspired by the Throwies \url{https://www.youtube.com/watch?v=AkgeeqakB7Y}

\section{Kit}
\begin{itemize}
    \item Multicolour Slow Change LEDs for example \url{https://bit.ly/4dQfCiF}
    \item CR2450 or other large coin batteries ideal 3v \url{https://bit.ly/4dWBetu} or CR2032.
    \item small magnets for example \url{https://bit.ly/4fchP8X}
    \item sticky tape and scissors (optional)
\end{itemize}
\section{Stage1: Have you got it the right way round.}
\begin{itemize}
    \item Taking the battery and an LED. 
    \item Spread the LED's 'legs' on leg should be longer than the other
    \item Put the battery horizontally (with the + side facing downwards between the legs)
    \item experiment to find the relationship with the legs and the +ve side of the battery: What is the link?
\end{itemize}

\section{Stage2: Put a magnet on it}
Now foor the fun bit. put a magnet on one of the legs to hold it the leg to the baterry - it would probabbly be ok with out it. Put now we have something that still to certain types of metals - which metals?

\section{Now what?}
It is up to you really.
\begin{itemize}
    \item Which of the legs is called the cathode and which is the anode. Cathode is the positive and Anode is the negative one. The cathode will let the LED light up if it is touching the positive side of the battery and the anode touches the negative side of the battery. which leg of the LED is the cathode (so the other is the anode)?
    \item How could we use these to make some wearable jewellery?
    \item If I want to create a letter A on flat surface what do I need and how would I do it.
\end{itemize}





